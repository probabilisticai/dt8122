\documentclass[a4paper, 12pt]{article}
\usepackage{fontspec}
\usepackage{polyglossia}
\setdefaultlanguage{english}
\usepackage{styles/ntnu}
\usepackage{amsmath}
\usepackage{amssymb}
\usepackage{latexsym}
\usepackage{graphicx}
\usepackage{float}
\usepackage{subfigure}
\usepackage{enumerate}
\usepackage[dvips]{color}
\usepackage{url}


%%% DOCUMENT ---------------------------------------------------------------------------------------

\begin{document}
\title{Probabilistic AI}{DT8122 — \textbf{Project Assignment - Uncertainty Quantification}}
\author{\textcolor{blue}{\href{mailto:dt8122@idi.ntnu.no}{dt8122@idi.ntnu.no}}}
\begin{center}
\vspace{-0.5cm}
Summer 2019
\end{center}

%---------------------------------------------------------------------------------------------------

\section{Introduction}
\label{sec:Introduction}

Uncertainty modeling and quantification is relevant in many real world applications. One possible
approach to this task is to use Bayesian methods in order to obtain the probability density of the
quantities of interest and use the obtained full densities to compute other quantities of interest
such as \textit{prediction
intervals}\footnote{\url{https://en.wikipedia.org/wiki/Prediction_interval}}. Traditionally, we
would resort to linear models in order to have a computationally feasible solution to the inference
problem. However, with the recent advances in Probabilistic AI, we are able to extend the space of
possible models with non-linear models. The objective of this project assignment is to explore some
of those techniques in a set of prediction tasks.

\section{Task}
\label{sec:Task}

The task concerns generation of prediction intervals (PIs) along with point predictions by applying
neural networks for quantifying uncertainty in regression tasks.

First, we ask you to choose and implement two recent deep generative modeling techniques from
Section~\ref{sec:Models}. Second, we ask you to identify shortcomings or limitations of the
implemented techniques. Finally, we ask you to implement some possible improvements (see
Section~\ref{sec:Research}).

Each part should be accompanied by empirical evaluation (see Section~\ref{sec:Evaluation}) using the
provided datasets (see Section~\ref{sec:Datasets}) and discussion of the results. This all compiled
into a final project report (max. 10 pages) accompanied with the code to reproduce your results.

\subsection{Models}
\label{sec:Models}

You are expected to implement a deep probabilistic regression model for uncertainty estimation and
use this model to generate prediction intervals and point predictions. For that you have to choose
two of the following techniques:

\begin{itemize}
 \item Dropout as a Bayesian Approximation~\cite{pmlr-v48-gal16},
 \item Bayes by Backprop~\cite{pmlr-v37-blundell15},
 \item Multiplicative Normalizing Flows~\cite{pmlr-v70-louizos17a},
 \item Bayes by Hypernet~\cite{krueger2017bayesian,DBLP:journals/corr/abs-1711-01297}.
\end{itemize}


\subsection{Research of Shortcomings, Limitations and Possible Improvements}
\label{sec:Research}

Given you have implemented the two selected approaches, discuss their respective shortcomings or
limitations and implement at least one possible improvement. Compare with previous results and
report your findings.

\subsection{Datasets}
\label{sec:Datasets}

The provided datasets are the same as in Gal and Ghahramani~\cite{pmlr-v48-gal16} and Pearce et
al.~\cite{pmlr-v80-pearce18a}. All datasets have target values ($y$) placed in the last column.

\subsection{Evaluation}
\label{sec:Evaluation}

For evaluation of your models use the tail 10\% of a dataset as a test set. Measure at least root
mean squared error (RMSE) for point predictions, and prediction interval coverage probability (PICP)
and mean prediction interval width (MPIW) for 95\% prediction intervals.

Following the notation from Pearce et al.~\cite{pmlr-v80-pearce18a},  let $\mathbf{x}_{i} \in
\mathbb{R}^{D}$ be the $i$th $D$ dimensional input features corresponding to target observation
$y_i$, where $1 \leq i \leq n$ for $n$ data points.
The predicted lower and upper bounds of PI are denoted by $\hat{y}_{Li}$ and $\hat{y}_{Ui}$. The PI
is then defined as
%
\begin{equation*}
P\left(\hat{y}_{L i} \leq y_{i} \leq \hat{y}_{U i}\right) \geq \gamma,
\end{equation*}
%
where $\gamma$ is $0.95$ for 95\% prediction interval.

\noindent
PICP is calculated as following:
%
\begin{equation*}
k_i =
\begin{cases}
  1 & \text{if}\ \hat{y}_{Li} \leq y_{i} \leq \hat{y}_{Ui}; \\
  0 & \text{otherwise},
\end{cases}
\end{equation*}
%
\begin{equation*}
c = \sum_{i=1}^{n} k_i,
\end{equation*}
%
\begin{equation*}
PICP = \frac{c}{n}.
\end{equation*}
%
\noindent
MPIW is calculated as following:
%
\begin{equation*}
MPIW = \frac{1}{n}\sum_{i=1}^{n} \hat{y}_{Ui} - \hat{y}_{Li}.
\end{equation*}

%---------------------------------------------------------------------------------------------------

\section{Submission Requirements}
\label{sec:SubmissionRequirements}

We expect you to submit the following:

\begin{itemize}
	\item {\bf Code} with your implementation in Python using Pyro or PyTorch of two of the four
	techniques of uncertainty modeling with neural networks.
	    \begin{itemize}
	        \item You can have the code in a Jupyter notebook, however you should make sure the code is
	        not dependent on a system specific configuration. To make sure that we can execute your
	        code, provide \texttt{requirements.txt} file, \textit{Dockerfile} or test it in Google
	        Colab\footnote{\url{https://colab.research.google.com}}.
	    \end{itemize}
	\item {\bf Report} that should include your model choices, inference methods, results from
	empirical evaluation, findings and discussion.
\end{itemize}

\noindent
All assignment artifacts are to be sent as a ZIP file to the email address \url{dt8122@idi.ntnu.no}.
The deadline is 12 July 2019 AoE (Anywhere on Earth).

\bibliographystyle{plain}
\bibliography{bibliography}

\end{document}

%%% END OF DOCUMENT --------------------------------------------------------------------------------
